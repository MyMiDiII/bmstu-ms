\chapter{Теоретическая часть}

\section{Формулы для вычисления величин}

Пусть $\vec{x} = (x_1, ..., x_n)$ --- выборка объема $n$ из генеральной
совокупности~$X$.

Максимальное и минимальное значения выборки:
\begin{align}\label{eq:01}
    M_{max} = x_{(n)},\\
    M_{min} = x_{(1)},
\end{align}

где $x_{(1)}, x_{(n)}$ --- крайние члены вариационного ряда, отвечающего
выборке~$\vec{x}$.

Размах выборки:

\begin{equation}\label{eq:03}
    R = M_{max} - M_{min}
\end{equation}

Оценки математического ожидания и дисперсии:

\begin{equation}\label{eq:04}
    \hat{\mu}(\vec{x}) = \frac{1}{n}\sum_{i=1}^{n} x_i
\end{equation}

\begin{equation}\label{eq:05}
    S^2(\vec x) = \frac 1{n-1} \sum_{i=1}^n (x_i-\overline x)^2,
\end{equation}

где $\overline x = \hat \mu$.

\section{Определение эмпирической плотности и гистограммы}

Если объем выборки достаточно велик ($n > 50$), то элементы выборки группируют в
так называемый интервальный статистический ряд. Для этого отрезок $J=[x_{(1)},
x_{(n)}]$ разбивают на $m$ равновеликих промежутков. Ширина каждого из них:

\begin{equation}\label{eq:06}
    \Delta = \frac{|J|}{m} = \frac{x_{(n)} - x_{(1)}}{m}.
\end{equation}

Далее получают:

\begin{equation}\label{eq:07}
    \begin{aligned}
        J_i &= [x_{(1)} + (i - 1) \Delta; x_{(1)} + i \Delta],
        & i = \overline{1, m-1}\\
        J_m &= [x_{(1)} + (m - 1) \Delta].
    \end{aligned}
\end{equation}

\textbf{Определение.} Интервальным статистическим рядом, отвечающим
выборке $\vec{x}$, называется таблица вида:

\begin{table}[htb]
    \centering
    \begin{tabular}{|c|c|c|c|c|}
        \hline
        $J_1$ & ... & $J_i$ & ... & $J_m$ \\
        \hline
        $n_1$ & ... & $n_i$ & ... & $n_m$ \\
        \hline
    \end{tabular}
\end{table}

Здесь $n_i$ --- число элементов выборки $\vec{x}$, попавших в промежуток $J_i$,
$i= \overline{1,m}$.

Также:

\begin{equation}\label{eq:08}
    \sum_{i=1}^{n} n_i = n;
\end{equation}

\begin{equation}\label{eq:09}
    m = [log_2 n] + 2.
\end{equation}

Пусть для данной выборки $\vec{x}$ построен интервальный статистический
ряд~$(J_i, n_i), i=\overline{1,m}$.

\textbf{Определение.} Эмпирической плотностью распределения (соответствующей
выборке $\vec{x}$) называется функция:

\begin{equation}
    \hat f(x) =
    \begin{cases}
        \frac{n_i}{n \Delta}, &x \in J_i, i = \overline{1; m} \\
        0, &\text{иначе} \\
    \end{cases}
\end{equation}

\textbf{Определение.} График эмпирической функции плотности называется
гистограммой.

\section{Определение эмпирической функции распределения}

Пусть $\vec{x} = (x_1, ..., x_n)$ --- выборка из генеральной совокупности $X$.

Обозначим $n(t, \vec{x})$ --- число компонент вектора $\vec{x}$,
которые меньше, чем $t$.

\textbf{Определение.} Эмпирической функцией распределения, построенной по
выборке $\vec{x}$, называют функцию $F_n: \mathbb{R} \to \mathbb{R}$,
определенную правилом: 

\begin{equation}
    F_n(t) = \frac{n(t, \vec x)}{n}
\end{equation}
