\chapter{Теоретическая часть}

\section{Определение $\gamma$-доверительного итервала для значения параметра
распределения случайной величины}

Пусть $X$ --- случайная величина, закон распределения которой известен с
точностью до неизвестного параметра $\theta$.

\textbf{Определение.} Интервальной оценкой параметра $\theta$ уровня $\gamma$
называется пара статистик $\underline \theta (\vec X)$ и $\overline \theta
(\vec X)$ таких, что

$$ P\{\theta \in (\underline \theta (\vec X), \overline \theta (\vec X))\} = \gamma.$$

\textbf{Определение.} $\gamma$-доверительным интервалом для параметра $\theta$
называется реализация (выборочное значение) интервальной оценки уровня $\gamma$
для этого параметра, то есть интервал $(\underline \theta (\vec x), \overline
\theta (\vec x))$ с детерминированными границами.

\section{Формулы для вычисления границ $\gamma$-доверительного интервала
для математического ожидания и дисперсии нормальной случайно величины}

Пусть $X \sim N(\mu, \sigma^2)$, где $\mu$ и $\sigma^2$ --- неизвестны.

Тогда для построения $\gamma$-доверительного интервала для $\mu$ используется
центральная статистика

$$g(\vec X, \mu) = \frac{\mu - \overline X}{S(\vec X)} \sqrt n \sim St(n - 1),$$

\noindentи границы $\gamma$-доверительного интервала для $\mu$ вычисляются по
формулам:

$$\underline \mu (\vec X) = \overline X - \frac{S(\vec X)t^{(n-1)}_{\frac{1 +
\gamma}{2}}}{\sqrt n},$$

$$\overline \mu (\vec X) = \overline X + \frac{S(\vec X)t^{(n-1)}_{\frac{1 +
\gamma}{2}}}{\sqrt n},$$

\noindentгде $\overline X = \frac{1}{n} \sum\limits_{i=1}^{n} X_i$,

\noindent~~~~~$S(\vec X) = \sqrt{\frac{1}{n-1} \sum\limits_{i=1}^{n} (X_i -
\overline X)^2}$,

\noindent~~~~~$t_{\frac{1+\gamma}{2}}^{(n-1)}$ --- квантиль уровня $\frac{1+\gamma}{2}$
распределения Стьюдента с \mbox{$n-1$~степенями} свободы,

\noindent~~~~~$n$ --- объем выборки.

Для построения $\gamma$-доверительного интервала для $\sigma^2$ используется
центральная статистика

$$g(\vec X, \sigma^2) = \frac{(n-1)S^2(\vec X)}{\sigma^2} \sim \chi^2(n - 1),$$

\noindentи границы $\gamma$-доверительного интервала для $\sigma^2$ вычисляются по
формулам:

$$\underline \sigma^2 (\vec X) = \frac{(n-1)S^2(\vec
X)}{h_{\frac{1+\gamma}{2}}^{(n-1)}},$$

$$\overline \sigma^2 (\vec X) = \frac{(n-1)S^2(\vec
X)}{h_{\frac{1-\gamma}{2}}^{(n-1)}},$$

\noindentгде $n$ --- объем выборки,

\noindent~~~~~$S^2(\vec X) = \frac{1}{n-1} \sum\limits_{i=1}^{n} (X_i -
\overline X)^2$,

\noindent~~~~~$h_{\frac{1+\gamma}{2}}^{(n-1)}$ и
$h_{\frac{1-\gamma}{2}}^{(n-1)}$ --- квантили уровня $\frac{1+\gamma}{2}$ и
$\frac{1-\gamma}{2}$ соответственно распределения хи-квадрат с
\mbox{$n-1$~степенями} свободы.

