\chapter{Задание}

\textbf{Цель работы}: построение доверительных интервалов для математического
ожидания и дисперсии нормальной случайной величины.

\textbf{Содержание работы}

\begin{enumerate}
    \item Для выборки объема $n$ из генеральной совокупности $X$ реализовать в
        виде программы на ЭВМ
        \begin{enumerate}[label=\asbuk*)]
            \item вычисление точечных оценок $\hat \mu(\vec x_n)$ и $S^2(\vec
                x_n)$ математического ожидания $\mathrm{M}X$ и дисперсии
                $\mathrm{D}X$ соответственно;
            \item вычисление нижней и верхней границ $\underline \mu (\vec
                x_n)$, $\overline \mu (\vec x_n)$ для
                \mbox{$\gamma$-доверительного} интервала для математического
                ожидания $\mathrm{M}X$;
            \item вычисление нижней и верхней границ $\underline \sigma^2 (\vec
                x_n)$, $\overline \sigma^2 (\vec x_n)$ для
                \mbox{$\gamma$-доверительного} интервала для дисперсии
                $\mathrm{D}X$.
        \end{enumerate}
    \item вычислить $\hat \mu(\vec x_n)$ и $S^2(\vec x_n)$ для выборки из
        индивидуального варианта;
    \item для заданного пользователем уровня доверия $\gamma$ и $N$ --- объема
        выборки индивидуального варианта:
        \begin{enumerate}[label=\asbuk*)]
            \item на координатной плоскости $Oyn$ построить прямую $y=\hat \mu
                (\vec x_N)$, также графики функций $y=\hat \mu (\vec x_n)$,
               $y=\underline \mu (\vec x_n)$ и $y=\overline \mu (\vec x_n)$
               как функций объема $n$ выборки, где $n$ изменяется от $1$
               до $N$.
            \item на другой координатной плоскости $Ozn$ построить прямую
                \mbox{$z=S^2 (\vec x_N)$}, также графики функций $z=S^2 (\vec
                x_n)$, $z=\underline \sigma^2 (\vec x_n)$ и \mbox{$z=\overline
                \sigma^2 (\vec x_n)$} как функций объема $n$ выборки, где $n$
                изменяется от $1$ до $N$.
        \end{enumerate}

\end{enumerate}

\textbf{Содержание отчета}

\begin{enumerate}
    \item определение $\gamma$-доверительного интервала для
        значения параметра распределения случайной величины;
    \item формулы для вычисления границ $\gamma$-доверительного интервала
        для математического ожидания и дисперсии нормальной случайной
        величины;
    \item текст программы;
    \item результаты расчетов и графики для выборки из индивидуального варианта
        (при построении графиков принять $\gamma = 0.9$).
\end{enumerate}
